\section{Progress}

Our project has made substantial progress in the weeks since we submitted our proposal.
We first made an effort to obtain accurate, high-quality data on the courses offered next semester.
We determined that the Schedule of Classes page was most likely to offer up-to-date data in a 
format easily analyzed by a machine. Once the course list for Spring 2014 was published, we used
a combination of the Schedule of Classes data and the ScottyLabs scheduling API in order to build
our own database of course information. Once we obtained this data, we built a bare-bones frontend
that allows users to search for courses by entering fragments of names and descriptions into search
boxes. We also added a secure registration and login system to ensure that users can only affect
their own data.

We have achieved the progress we've made so far largely by following the methodology described in 
our proposal. We split the work into ``frontend'' and ``backend'' sections, as described in our
Gantt chart, and met weekly in order to ensure that we remained in sync. These meetings focused
mainly on the design of the interface between the frontend and the backend, in order to ensure
that no member of either half would be unable to work due to not knowing what the plan for the
other half was. We also met in smaller, 2-person groups in order to make more specific internal
design decisions, in particular in the design of the automatic scheduling algorithm.

We have split the work largely by having each member work on the things most interesting to them.
Maxime has led the backend code-writing by producing the data and implementing the existing
authentication and search systems. Joshua has spent significant time designing and writing a
useful, stable API in order to make future extensions of the existing codebase fast and easy.
Chris and Jacob have spent most of their time designing the eventual user interface of the project,
coming up with mock-ups and backend requirements in order to ensure no work is wasted. This separation
of concerns has allowed each member to make significant progress without needing to wait for
the others, allowing development to proceed apace.