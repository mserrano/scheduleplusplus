\section{Plan}
Our goal is to create a fully functional, intuitive website that will allow students to design their 
schedule with the option of computer assistance. Our group will implement, evaluate and write a final
report on this project. We will use our understanding of web crawling, scheduling algorithms, web design,
and other technologies to implement an efficient, effective solution to the problem of automatically
scheduling classes and displaying informative schedules. The Methodology subsection 
of the Approach section in this proposal provides more details of our implementation.
As we develop the website, we plan to test the website and its functionality on a variety
of web browsers, ensuring that it functions correctly and properly for as many users as possible.
We will also stress-test the performance of the website, to ensure that it can reliably serve
users even when under heavy load, and that it does not begin to produce erroneous results.
The Evaluation section of this proposal describes in more complete detail how we will measure
the success of our project. Last, we will conclude our work with a final report and presentation of our results.

In order to efficiently and effectively complete this project, our group plans to split the work among
our members based upon each member's past experience with various technologies. We will work
to implement this project through October and November. The Project Schedule subsection 
of the Approach section provides a more detailed overview of our schedule.
We will meet at least once a week to collaborate on any problems we encounter, as well
as to keep each other up to date on work that has been done during the week. We will
use Internet Relay Chat to communicate outside of these meetings, and use Git to
keep track of our software's versions as well as distribute code and documents within
the group.